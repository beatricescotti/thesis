\addcontentsline{toc}{chapter}{Abstract}
\chapter*{Abstract}


Early diagnosis is the most effective weapon in the fight against cancer, with computed tomography and medical imaging playing a crucial role in this process. Recently, artificial intelligence has shown great potential to identify and classify tumor lesions in their early stages. However, the application of these techniques faces a fundamental limitation: the scarcity of annotated data available for training.

As is well known in machine learning, the quality of results depends strictly on the quality and completeness of the training data ("garbage in, garbage out"). To overcome this limitation, this thesis explores the use of transfer learning, a technique that allows the transfer of knowledge from a data-rich source domain to a data-limited target domain.

Specifically, the research focuses on the use of simulated images of high-energy particle jets, constructed specifically for this study, characterized by two distinct classes of physical phenomena: a noisy background with physical characteristics similar to the main jets and morphological properties similar to tumor structures.

The proposed approach involves a pretraining phase on a large high-energy physics dataset, transferring the learned weights to the medical domain, and then fine-tuning them on limited clinical datasets. This method aims to leverage the feature extraction capabilities developed in the particle physics domain, adapting them to the medical context where annotated data are scarce, but diagnostic accuracy is crucial.