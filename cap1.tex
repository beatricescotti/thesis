\chapter{Computed Tomography}
Computed Tomography (CT) is a medical imaging technique that uses X-rays to generate detailed cross-sectional images of the human body. Since its introduction in the 1970s, CT has revolutionized diagnostic radiology by enabling the visualization of internal structures with high spatial resolution and three-dimensional reconstruction capabilities. Unlike conventional radiography, which produces a single projection image, CT acquires multiple X-ray measurements from different angles around the patient and uses computer algorithms to reconstruct the corresponding anatomical slices.

\begin{figure}[ht]
    \centering
    \includegraphics[width=0.7\textwidth]{immagini/cap1/rad_ct1.png} 
    \caption{On the left (figure a.) we can see a chest radioraphy, on the right (figure b.) we can see a CT scan of the same area. The CT scan provides a much more detailed view of the internal structures, allowing for better diagnosis and treatment planning.}
    \label{fig:ct_scanner}
\end{figure}

CT plays a crucial role in clinical practice due to its ability to provide fast, accurate, and non-invasive diagnostic information for a wide range of applications. Modern CT scanners are equipped with advanced technologies that significantly improve image quality, reduce scan time, and minimize radiation dose.


This chapter provides an overview of the main aspects of CT imaging that are relevant for medical physics applications. The first part briefly describes the experimental setup of a CT scanner, the second part focuses on the mathematical principles behind image reconstruction, highlighting the algorithms used to convert projection data into tomographic slices. Finally, the chapter introduces the concept of image post-processing, with particular attention to windowing and its impact on image interpretation.

\section{Tomography setup}
The basic idea behind tomography is to obtain a three-dimensional image of an object from a series of two-dimensional images acquired from different angles.
For this purpose, a CT scanner consists, like any radiation measurement system, of a radiation source (an X-ray tube) and an array of particle detectors. To acquire projections from multiple angles, the X-ray tube and the detectors are mounted on a rotating gantry, which allows the system to rotate around the patient or, more generally, around the object being examined.

\begin{figure}[ht]
    \centering
    \includegraphics[width=0.45\textwidth]{immagini/cap1/CT.png} 
    \caption{} 
    \label{fig:tomografia}
\end{figure}

As shown in Figure \ref{fig:tomografia}, the detectors are not arranged on a plane perpendicular to the X-ray source, but rather along a circular arc. This configuration ensures that all pixels in the projection correspond to the same physical size, thus avoiding geometric distortions in the reconstructed image.
Depending on the scanner design, the acquisition can be performed in a step-and-shoot mode or using continuous rotation, as in helical (spiral) CT. The X-ray beam is collimated into a fan or cone shape to cover the entire detector array. Each detector element typically consists of a scintillator coupled to a photodiode, which converts the X-ray photons into an electrical signal proportional to the received dose. Synchronizing the gantry rotation with data acquisition allows the collection of a sufficient number of projections to ensure accurate image reconstruction.
\begin{figure}[!ht]
    \centering
    \includegraphics[width=0.7\textwidth]{immagini/cap1/setup.png} 
    \caption{In the frontal view we can see, in order, the X-ray tube (radiation source), the rotating gantry (support for the X-ray tube and detectors), the field of measurement (where the patient or the object is allocated), the anti scatter collimators (to reduce the amount of scattered radiation reaching the detectors), and the detectors (which measure the intensity of the X-rays after they pass through the object). In the lateral view we can see the presence of colimators, used to adjust the narrowness of the beam depending on the type of scan and the area of interest.} 
    \label{fig:CT_detector}
\end{figure}

\section{Image Reconstruction}
Una radiografia è una proiezione bidimensionale della radizione che attraversa un oggetto tridimensionale, il che mi dà diverse ambiguità se voglio ricostruire il vero profilo dell'oggetto.
Una tomografia è invece un insieme di proiezioni dell'oggetto ottenute a diverse angolazioni.
Fare una tomografia, quindi, consiste nel misurare molti integrali di radiazione lungo diverse direzioni, per poi ricostruire l'immagine tridimensionale dell'oggetto in questione.

\subsection{Mathematical Implementation}
The mathematical model of a tomographic reconstruction is based on the Radon transform, which describes how the projection of a function (the object to be reconstructed) can be obtained by integrating the function along straight lines.
Given the function \( f(x,y) \), that represents the 2D density of the object to be reconstructed and a certain projection at the angle \( \theta \), the path of each X-ray can be described by the straight line:
\begin{equation}
    x \cos(\theta) + y \sin(\theta) = s
\end{equation}
where \( s \) is the distance from the origin to the line along which the X-ray is traveling. The CT measurement corresponds to an integration of the funciotn \( f(x,y) \) along such a ray:
\begin{equation}
    p(s,\theta) = \int_{-\infty}^{+\infty} f(x,y) \delta(x \cos(\theta) + y \sin(\theta) - s) dx dy
\end{equation}
where \( p(s,\theta) \) is the projection of the function \( f(x,y) \), and the Dirac delta function \( \delta(x \cos(\theta) + y \sin(\theta) - s) \) ensures that the integration is performed only along the line defined by the angle \( \theta \) and the distance \( s \).
We can now define the \textbf{Radon transform \( R \)} as the 1D projection (or radiograph) of the 2D function \( f(x,y) \) at the angle \( \theta \).
\begin{equation}
    P = R\cdot f
\end{equation}
\begin{figure}[h!]
    \centering
    \includegraphics[width=0.5\textwidth]{immagini/cap1/projection.png} 
    \caption{} 
    \label{fig:radon}
\end{figure}

Now, if we take the 1D Fourier transform \( P(\theta, u) \) of the projection data \( p(s, \theta) \):
\begin{equation}
    P(\theta, u) = \int_{-\infty}^{+\infty} p(s, \theta) e^{-2\pi i u s} ds = \int \int dx dy f(x,y) e^{2\pi i u (x \cos\theta + y \sin\theta)}
\end{equation}

and we compare to the general 2D Fourier transform of the function \( f(x,y) \):
\begin{equation}
    F(u_x, u_y) = \int \int dx dy f(x,y) e^{-2\pi i (u_x x + u_y y)}
\end{equation}
where \( u_x = u \cos\theta \) and \( u_y = u \sin\theta \), we can see that the Radon transform is related to the 2D Fourier transform by the following relationship:
\begin{equation}
    P(\theta, u) = F(u \cos\theta, u \sin\theta)
\end{equation}
This means that 1D fourier transform of the projection $Rf(s, \theta)$ is equal to the 2D fourier transform of $f(x,y)$ in (\( u_x, u_y \)) coordinates, where \( u_x = u \cos\theta \) and \( u_y = u \sin\theta \).

To calculate the original function \( f(x,y) \), let's do the Fourier antitransform: 
\begin{equation}
    f(x,y) = \int \int du_x du_y F(u_x, u_y) e^{2\pi i (u_x x + u_y y)}
\end{equation}
Now we can substitute \( u_x = u \cos\theta \) and \( u_y = u \sin\theta \) to finally obtain:
\begin{equation}
    f(x,y) = \int_{0}^{\pi} d\theta \int_{-\infty}^{+\infty} du |u| P(\theta, u) e^{2\pi i u (x \cos\theta + y \sin\theta)} 
\end{equation}
the absolute value of \( u \) is the jacobian of the transformation: \( du_x du_y = |u| d\theta du \) .
\begin{figure}[h!]
    \centering
    \includegraphics[width=0.8\textwidth]{immagini/cap1/radon.png} 
    \caption{The Radon transform is a mathematical operation that transforms a function defined in the spatial domain into a function defined in the projection domain. It is used in computed tomography to reconstruct images from projection data.} 
    \label{fig:radon_transform}
\end{figure}

The absolute value of $u$ mathematically is the Jacobian of the coordinate transformation, but has also a physical meaning, because is filtering the high frequency noise from the image, being a low-pass filter.
The Radon transform with the absolute value of \( u \) is called the \textbf{filtered back projection} (FBP) and is the most common method for reconstructing images from CT data.


\section{Windowing}
In computed tomography (CT), the intensity values of the pixels are quantified in \textbf{Hounsfield units (HU)}, a standardized scale that reflects the radiodensity of the tissue. This scale is defined relative to the attenuation coefficients of water and air:
\begin{equation}
    HU_{tissue} = 1000 \times \dfrac{\mu_{tissue} - \mu_{water}}{\mu_{water}-\mu_{air}}
\end{equation}
where $\mu$ represents the linear attenuation coefficient. Key reference points: water = 0 HU for definition, air = -1000 HU. \\
\textbf{Windowing} is a technique used in computed tomography (CT) to \textbf{enhance the visibility of specific tissues or structures within the body.} It involves adjusting the range of Hounsfield units (HU) displayed in the CT image, allowing radiologists to focus on particular types of tissues, such as bones, soft tissues, or air-filled spaces. The window width and center determine the contrast and brightness of the image, respectively.
\begin{itemize}
    \item \textbf{Window width} (WW): This parameter controls the range of Hounsfield units displayed in the image. A narrow window width enhances the contrast between tissues with similar densities, while a wider window width allows a broader range of densities to be visualized.
    \item \textbf{Window center} (WC): This parameter sets the midpoint of the Hounsfield unit range displayed in the image. Adjusting the window level changes the brightness of the image, allowing radiologists to focus on specific tissue types.
\end{itemize}
The choice of window width and level depends on the specific clinical question and the type of tissue being examined. 

\begin{figure}
    \centering
    \includegraphics[width=0.8\textwidth]{immagini/other_windowing.png}
    \caption{CT Hounsfield Unit (HU) ranges for key tissues. Highlighted regions show typical HU values for diagnostic reference, from dense compact bone ($\sim 3000$ HU ) to air ($1000$ HU), with soft tissues ($20-80$ HU) and fluids (blood $\sim 45$ HU, water $0$ HU) in intermediate ranges.}
    \label{fig:general_windowing}
\end{figure}

\begin{figure}
    \centering
    \includegraphics[width=0.75\textwidth]{immagini/windowing.png}
    \caption{Standard CT windowing settings for different anatomical structures. Full CT value range ([-1000, 3000] HU). Clinical presets with window center (WC) and width (WW) values for bone (WC=1000, WW=2500), mediastinum (WC=-50, WW=400), and lung (WC=-600, WW=1700) visualization.}
    \label{fig:windowing_lung}
\end{figure}



\section{CT planes}
In computed tomography (CT), the images are typically acquired in three orthogonal planes: axial, coronal, and sagittal. Each plane provides a different perspective of the anatomical structures within the body.
\begin{itemize}
    \item \textbf{Axial}: horizontal plane that divides the body into upper and lower parts.
    \item \textbf{Coronal}: vertical plane that divides the body into anterior (front) and posterior (back) parts.
    \item \textbf{Sagittal}: vertical plane that divides the body into left and right parts.
\end{itemize}

\begin{figure}[h!]
    \centering
    \includegraphics[width=0.6\textwidth]{immagini/cap1/planes.jpg}
    \caption{CT planes: axial, coronal, and sagittal. The axial plane is horizontal, the coronal plane is vertical and divides the body into front and back, and the sagittal plane is vertical and divides the body into left and right.}
    \label{fig:ct_planes}
\end{figure}
